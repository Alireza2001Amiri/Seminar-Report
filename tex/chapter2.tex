% !TeX root=../main.tex
\chapter{مروری بر مطالعات انجام شده}
%\thispagestyle{empty} 
\section{مقدمه}
در این فصل، پژوهش‌های پیشین در زمینه‌ی موتورهای مسطح مبتنی بر شناوری مغناطیسی (MLPM) با تمرکز بر ویژگی‌های اساسی آنان که به طور کلی در بخش‌های زیر دسته‌بندی شده‌اند، مورد بررسی قرار می‌گیرند. 
\begin{itemize}
	\item
		\textit{معماری دستگاه}:
بررسی انواع معماری‌های موجود برای MLPM و تأثیر آن‌ها بر عملکرد کلی سیستم.
	\item
		\textit{ساختار آهنرباهای دائمی و الکتریکی}:
مرور انواع آهنرباهای الکتریکی و چینش‌های مختلف آهنربا‌های دائمی و نقش آن‌ها در بهینه‌سازی عملکرد سیستم.
	\item
		\textit{طراحی کنترلر}:
معرفی روش‌های کنترل کلاسیک و مدرن برای این سیستم‌ها و چگونگی بهبود پایداری و دقت حرکت.
	\item
		\textit{روش‌های شناسایی سیستم و مدل‌سازی دینامیکی}:
تحلیل روش‌های شناسایی و تخمین مدل‌های دینامیکی سیستم برای شبیه‌سازی و بهینه‌سازی عملکرد.
\end{itemize}
در بخش‌های بعد، پژوهش‌های انجام‌شده بر اساس این ویژگی‌ها ارزیابی شده و مزایا و معایب هر روش مورد بررسی قرار می‌گیرد.

\section{معماری دستگاه‌های MLPM}
سیستم‌های شناوری مغناطیسی به دلیل ماهیت ناپایدارشان بدون استفاده از حلقه‌های کنترلی نمی‌توانند پایداری لازم را فراهم کنند. به همین دلیل، در تمامی ساختارهای پیشنهادی، از سیم‌پیچ‌های الکتریکی برای تولید میدان مغناطیسی با شدت کنترل ‌شده استفاده می‌شود. این سیم‌پیچ‌ها وظیفه دارند تا موقعیت جسم معلق را پایدار کرده و آن را در حالت مطلوب نگه ‌دارند.

در طراحی موتورهای مسطح، که از دو بخش ثابت
\LTRfootnote{Stator}
 و متحرک
\LTRfootnote{Mover}
تشکیل شده‌اند، امکان تغییر در طراحی و محل قرارگیری آهنرباهای الکتریکی و دائمی وجود دارد. نیروی مغناطیسی وارد بر بخش متحرک می‌تواند به‌صورت جاذبه‌ای از بالا یا دافعه‌ای از پایین اعمال شود. با این حال، در موتورهای مسطح به دلیل لزوم کم بودن فاصله میان سیم‌پیچ‌ها و اجسام معلق، اعمال نیروی جاذبه‌ای از بالا امکان‌پذیر نیست. به همین دلیل، در تمامی طراحی‌ها، نیروی مغناطیسی دافعه‌ای از سمت پایین به بخش متحرک وارد می‌شود که امکان جابه‌جایی اجسامی که بر روی آنها قرار می‌گیرند را فراهم می‌کند.

با توجه به این موارد، دو طراحی کلی برای ساخت دستگاه‌های MLPM ارائه می‌شود که در ادامه به بررسی هر یک پرداخته خواهد شد.


\section{مروری بر ادبیات موضوع}
در این قسمت باید به کارهای مشابه دیگران در گذشته اشاره کرد و وزن بیشتر این قسمت بهتر است به مقالات ژورنالی سال‌های اخیر (۲ تا ۳ سال) تخصیص داده شود. به نتایج کارهای دیگران با ذکر دقیق مراجع باید اشاره شده و جایگاه و تفاوت تحقیق شما نیز با کارهای دیگران مشخص شود. استفاده از مقالات ژورنال‌های معتبر در دو یا سه سال اخیر، می‌تواند به اعتبار کار شما بیافزاید.

.
.
.
.
.
.
.


\section{نتیجه‌گیری}
‌در نتیجه‌گیری آخر این فصل، با توجه به بررسی انجام شده بر روی مراجع تحقیق، بخش‌های قابل گسترش و تحقیق در آن حیطه و چشم‌اندازهای تحقیق مورد بررسی قرار می‌گیرند.	در برخی از تحقیقات، نتیجه نهایی فصل روش تحقیق، ارائهٔ یک چارچوب کار تحقیقی 
\lr{(research framework)}
است.
.
.
.
.
.