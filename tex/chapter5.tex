% !TeX root=../main.tex
\chapter{بحث و نتیجه‌گیری}
%\thispagestyle{empty} 


این گزارش به‌صورت جامع به بررسی طراحی و ساخت موتورهای مسطح مبتنی بر شناوری مغناطیسی (MLPM) پرداخته و جنبه‌های مختلف این سیستم‌ها شامل معماری، طراحی آهنرباهای دائمی، سیستم‌های کنترلی و روش‌های مدل‌سازی را مورد تحلیل قرار داده است. هدف اصلی این تحلیل‌ها، شناسایی بهترین روش‌ها و ارائه راهکارهایی برای بهبود عملکرد این دستگاه‌ها بر اساس پژوهش‌های پیشین بوده است.
در معماری دستگاه، بررسی‌ها نشان داد که استفاده از سیم‌پیچ‌های متحرک و آهنرباهای ثابت به دلیل محدودیت‌های ذاتی مانند مشکلات مرتبط با اتصالات الکتریکی و خنک‌کاری سیم‌پیچ‌ها، راهکاری با کارایی کمتر محسوب می‌شود. در مقابل، استفاده از سیم‌پیچ‌های ثابت و آهنرباهای دائمی متحرک، به دلیل حذف محدودیت‌های فوق و بهبود عملکرد حرکتی بخش متحرک، به‌عنوان معماری بهینه و مناسب‌تر برای کاربردهای MLPM معرفی شد.
در طراحی آهنرباهای دائمی، مقایسه بین آهنرباهای دیسکی و آرایه‌های هالباخ نشان داد که آرایه‌های هالباخ به‌ویژه در چینش‌های یک‌بعدی و دوبعدی، عملکرد بهتری از نظر تقویت میدان مغناطیسی دارند. این آرایه‌ها، از طریق خنثی کردن میدان در یک سمت و تقویت آن در سمت دیگر، قادر به تولید میدان مغناطیسی قوی‌تری هستند که امکان کنترل دقیق‌تر نیروها و جابه‌جایی‌ها را فراهم می‌آورد. آهنرباهای دیسکی هرچند از نظر طراحی ساده‌تر هستند، اما به دلیل ناپایداری و غیر یکنواختی میدان مغناطیسی، کارایی کمتری در سیستم‌های MLPM دارند. آرایه‌های دوبعدی هالباخ، هرچند مزیت‌های بسیاری در تقویت میدان مغناطیسی دارند، با چالش‌هایی همچون ایجاد نوسانات بیشتر در میدان همراه هستند که باید با استفاده از طراحی دقیق‌تر مدیریت شود.
در حوزه کنترلرها، کنترلرهای کلاسیک نظیر PID به دلیل سادگی و کارایی اثبات‌شده، همچنان به‌عنوان گزینه‌ای مناسب برای سیستم‌های MLPM مطرح هستند. با این حال، نتایج پژوهش‌ها نشان می‌دهد که در صورتی که اطلاعات دینامیکی سیستم در دسترس باشد، کنترلرهای مبتنی بر مدل پیش‌بینی (MPC) و یا کنترلرهای مبتنی بر هوش مصنوعی همچون GRU، به دلیل توانایی پیش‌بینی رفتار سیستم و اعمال کنترل دقیق‌تر، از عملکرد بهتری برخوردارند. این روش‌های پیشرفته می‌توانند پایداری سیستم را افزایش داده و خطاهای ناشی از کنترل را کاهش دهند، به‌ویژه در کاربردهایی که نیاز به دقت بالا دارند.

